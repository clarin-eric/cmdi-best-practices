\subsection{Workflow}\label{authoring-workflow}

The workflow for creation of CMDI records for hosted resources will
depend on the specific conditions and requirements at your centre. CMDI
files can be created (semi-)manually (by editing XML files or using
tools such as CMDI Maker,\footnote{See http://cmdi-maker.uni-koeln.de/,
  but notice that CMDI Maker only supports 2 profiles!}
COMEDI,\footnote{See http://clarino.uib.no/comedi/} and Arbil\footnote{See
  https://tla.mpi.nl/tools/tla-tools/arbil/, but notice that Arbil is
  not maintained anymore!} - this will usually be done by the resource
owners), or by automatic conversion from existing metadata formats with
the usual benefits and disadvantages of manual and automatic data
creation respectively.

Concerning metadata quality assessment in the workflow, some aspects can
be checked before CMDI is imported into the VLO:

\begin{itemize}
\tightlist
\item
  all CMDI metadata must validate against the profile XSD
\item
  manually created CMDI should go through automatic consistency checks
\item
  all CMDI metadata should be checked for features that cannot be
  covered by the profile XSD - the
  \href{https://github.com/clarin-eric/cmdi-instance-validator/releases/latest}{CMDI
  Validator} includes a set of Schematron rules that can be used for
  additional specific Schematron checks (and adapted as necessary)
\item
  the \href{https://clarin.oeaw.ac.at/curate/}{CLARIN Curation Module}
  should be used to asses records
\end{itemize}

The following aspects on the other hand need to be checked after VLO
import:

\begin{itemize}
\tightlist
\item
  appropriateness of all normalised and mapped values for VLO facets
  (cf.~VLO-mapping) - any mapping issues should be reported via
  \href{mailto:cmdi@clarin.eu}{\nolinkurl{cmdi@clarin.eu}}
\item
  appropriateness of the display of resource hierarchies, especially in
  cases where MdSelfLink and/or ResourceProxy values are changed as part
  of the workflow.
\end{itemize}
