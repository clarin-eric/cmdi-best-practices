\section{Preface}
In 2016, both version 1.2\footnote{See \url{https://www.clarin.eu/cmdi1.2}} of
the \note{Alex: there was a link from Component Metadata to \url{https://www.gitbook.com/book/clarin-eric/cmdi-best-practices/edit\#} which doesn't seem to make much sense to me, so I removed it} Component Metadata (CMD) Infrastructure (CMDI) and a first complete technical specification (CE-2016-0880 \cite{ce20130095}) \note{Alex: If the spec is available online, we should add a link to it here} of this metadata standard were introduced. The new version introduced new possibilities, which have been gradually opened up by the ecosystem of tools and registries in CMDI. One of the key properties of CMDI is its flexibility, which makes it possible to create metadata records closely tailored to the requirements of resources and tools/services. However, design and implementation choices made at various levels in the CMD lifecycle might influence how well or easily a CMD record is processed and its associated resources made available in the CLARIN infrastructure. Knowledge on this has traditionally been scattered around in various documents, web pages and even completely hidden from sight in experts' minds. To make this
knowledge explicit, the CMDI and Metadata Curation Task Forces have teamed up to create this Best Practice guide. Hopefully this guide, together with the technical CMDI 1.2 specification \cite{ce20130095}, will be a valuable knowledge base and will help any (technical) CMDI user to bring her CMD records to their full potential use within \href{https://www.clarin.eu}{CLARIN}.

This guide keeps being improved. You can find the latest version of the guide at \url{https://www.clarin.eu/cmdi}.

\note{Alex: I think the guide wavers between American and British spelling throughout. We should decide for one of them and change it to be consistent.}
\note{Twan: +1, I would go for Brittish}
