\section{How to use this Guide}

Although the guide will express the best practices in strict wording, it is possible, and sometimes even allowed to break a best practice. To indicate the impact of breaking a best practice a priority level is given:

\begin{enumerate}
\def\labelenumi{\arabic{enumi}.}
\tightlist
\item
  \emph{High priority}: a CMD record not following this best practice severely counters the basic principles of CMDI \bptodo{TODO: reference}, e.g.~reusable components with explicit semantics, or common best practices for resources on the web, e.g.~URLs should be resolvable not dead links \bptodo{TODO: reference};
\item
  \emph{Medium priority}: default priority of a best practice, which means it can be broken if you have good local (technical or organizational) reasons to deviate;
\item
  \emph{Low priority}: a CMD record meeting this best practice is better adapted to the CLARIN infrastructure, but not following it will not have bad consequences.
\end{enumerate}

%\note{Menzo: basic principles of CMDI: 1) components, 2) generic as possible specific as needed (reusable), 3) explicit semantics}
Best practices with a \emph{high} priority should be followed. If a best practice with a medium level priority is not followed the reasons why should always be documented.

Compliance with some best practices can be (partially) assessed
automatically. CLARIN provides the following tools and services:

\begin{enumerate}
\def\labelenumi{\arabic{enumi}.}
\tightlist
\item
  CMDValidator:\footnote{See
    \url{https://github.com/clarin-eric/cmd-validate/releases}} The validator for CMD components and profiles is in general used via the Component Registry;
\item
  CMDI Instance Validator:\footnote{See
    \url{https://github.com/clarin-eric/cmdi-instance-validator/releases}}
  The corresponding validator for CMD records;
\item
  CLARIN curation module:\footnote{See
    \url{https://clarin.oeaw.ac.at/curate/}} This service provides insight into the quality of CMD profiles, records and endpoints, which can be accessed directly via their URLs.
\end{enumerate}

%The functionalities of the two more low level validators are being integrated into the CLARIN curation module to provide a single service for CMD quality assessment. %\note{Menzo: @Matej: is ``are being integrated into'' not a too strong claim?}

Some best practices have a direct relationship with the CLARIN B Centre requirements (\href{http://hdl.handle.net/11372/DOC-78}{CE-2013-0095}) \cite{ce20130095},
which is indicated next to the priority. A CLARIN B Centre will have to meet such a requirement and the best practice.
