\documentclass[]{article}
\usepackage{lmodern}
\usepackage{amssymb,amsmath}
\usepackage{ifxetex,ifluatex}
\usepackage{fixltx2e} % provides \textsubscript
\ifnum 0\ifxetex 1\fi\ifluatex 1\fi=0 % if pdftex
  \usepackage[T1]{fontenc}
  \usepackage[utf8]{inputenc}
\else % if luatex or xelatex
  \ifxetex
    \usepackage{mathspec}
  \else
    \usepackage{fontspec}
  \fi
  \defaultfontfeatures{Ligatures=TeX,Scale=MatchLowercase}
\fi
% use upquote if available, for straight quotes in verbatim environments
\IfFileExists{upquote.sty}{\usepackage{upquote}}{}
% use microtype if available
\IfFileExists{microtype.sty}{%
\usepackage{microtype}
\UseMicrotypeSet[protrusion]{basicmath} % disable protrusion for tt fonts
}{}
\usepackage{hyperref}
\hypersetup{unicode=true,
            pdfborder={0 0 0},
            breaklinks=true}
\urlstyle{same}  % don't use monospace font for urls
\IfFileExists{parskip.sty}{%
\usepackage{parskip}
}{% else
\setlength{\parindent}{0pt}
\setlength{\parskip}{6pt plus 2pt minus 1pt}
}
\setlength{\emergencystretch}{3em}  % prevent overfull lines
\providecommand{\tightlist}{%
  \setlength{\itemsep}{0pt}\setlength{\parskip}{0pt}}
\setcounter{secnumdepth}{0}
% Redefines (sub)paragraphs to behave more like sections
\ifx\paragraph\undefined\else
\let\oldparagraph\paragraph
\renewcommand{\paragraph}[1]{\oldparagraph{#1}\mbox{}}
\fi
\ifx\subparagraph\undefined\else
\let\oldsubparagraph\subparagraph
\renewcommand{\subparagraph}[1]{\oldsubparagraph{#1}\mbox{}}
\fi

\date{}

\begin{document}

\subsection{Optimising for
discoverability}\label{optimising-for-discoverability}

{[}See
\href{https://www.gitbook.com/book/cmdi-taskforce/cmdi-best-practices/changes/23}{CR
\#23}{]}

An important motivation for creating and publishing high quality
metadata for resources is that doing so can vastly increase the
discoverability of these resources. Metadata generally provides
information that is not (directly) available from the associated digital
object, service, tool or other resource itself, and therefore caters to
use cases complementary to \emph{content search}.

Most of the best practices described in this document serve the purpose,
either implicitly or explicitly, to improve the discoverability of the
described resources. Therefore in general it can be said that following
the described best practices as closely as possible is the best way to
optimise for discoverability. However in this section we will explicitly
assume the perspective of discoverability and focus on a number of
salient aspects that either have the potential of strongly affecting
discoverability, are a common cause of suboptimal discoverability, or
may not be obvious to the average metadata modeller or author.

A few examples of scenarios we consider cases of resource discovery and
believe should be recognised and considered in the context of metadata
modelling and authoring: \textgreater{} {[}sh{]} ``A few examples of
scenarios \textbf{which} we consider {[}\ldots{}{]} are listed below:'';
``Here we list a few examples \ldots{} which \ldots{}''

\begin{itemize}
\tightlist
\item
  A user visits a \textbf{specific repository} where they may or may not
  expect to find access to a certain resource, or a resource with
  certain properties. If successful, the desired resource is provided to
  the user through the platform that was used to initiate the discovery.
\item
  A user visits a metadata-driven catalogue, search engine or other
  \textbf{platform that aggregates metadata} from multiple sources, such
  as the \href{https://www.clarin.eu/vlo}{Virtual Language Observatory},
  and browses or queries the indexed records on basis of a more or less
  specific set of criteria, eventually reaching a location that provides
  (a path to) access to the actual resource, which may involve one or
  more intermediate steps including one or more interactions with the
  `native' repository (see above) that hosts the resource.
\item
  A user performs a search using a \textbf{general purpose search
  engine} (such as Google) that crawls web pages and other online
  content, using salient keywords that match the metadata and/or content
  of the resource. Results that lead the user to the resource may
  include the repository or another search or aggregation platform.
\item
  A resource is \textbf{referenced} in a paper, in online content such
  as a website or blog post, or included in a collection that is
  consumed by a user, and hence discovered. The reference has to be
  actionable in some way and provide direct or indirect access to the
  actual resource in order for the discovery process to be considered
  successful.
\end{itemize}

Metadata creators can choose to optimise for one or more types of
discovery scenarios.

\subsubsection{Metadata formats and
representation}\label{metadata-formats-and-representation}

By design, metadata aggregators such as the VLO have a more or less
\textbf{generic search interface and data model}, which means that in
most cases they will not be as well suited for searching for resources
in a specific context as search platforms tailored to that context.
I.e., repositories with a relatively limited scope in terms of content
can harness their `knowledge' about the contained information and its
shape to provide enhanced means of search, exploration and presentation.
In fact, many repository solutions have a data schema (encoded in their
database or other data storage facility) that is optimised for these
purposes. In those cases, CMDI or for example Dublin Core metadata is
often only made available (exported) as `secondary' form of
representation. In many cases this is an apt solution, that makes it
possible to combine the benefits of specificity and generality. However
it is also important that the exported metadata is not merely
\emph{available} but actually \textbf{fit for aggregation}, which
implies a different context than that of the `native' repository.
Factors to keep in mind when designing the export from an internal
representation to metadata for harvesting:

\subparagraph{Implicit vs explicit
context}\label{implicit-vs-explicit-context}

The meaning of abbreviations and ids or `relative' descriptors (e.g.
`Organisation: Linguistics department') might be understood or
resolvable within its native context but aggregation will make such
context far less obvious or even invisible.

\subparagraph{Custom vocabularies vs broadly adapted
vocabularies}\label{custom-vocabularies-vs-broadly-adapted-vocabularies}

Often tailor made vocabularies are used that allow for description and
classification of resources with a high level of accuracy and detail.
However in an aggregated context, the value of more generic but broadly
understood and supported vocabularies becomes apparent. Obviously the
information exposed by highly specific vocabularies don't have to be
sacrificed when exporting metadata, but one should consider extending
the metadata with additional values from commonly understood
vocabularies. For example, your metadata might contain detailed spatial
information in the form of geographic coordinates; in that case your
repository can probably process this information well but the same
cannot be assumed from any aggregator. A good strategy would be to also
export one or more levels of corresponding geographical common names
(e.g.~country, city).

\subparagraph{Nature of the targeted aggregation
platform(s)}\label{nature-of-the-targeted-aggregation-platforms}

As a metadata provider you will generally know which platform(s) will
aggregate your metadata, or else it is likely to be in your interest to
cater for one or more highly relevant aggregators, for example popular
domain aggregators. Find out which vocabularies are supported by these
aggregators, which information gets displayed saliently and how links
back to your resources and/or repository are achieved.

\subsubsection{Metadata completeness and VLO
facets}\label{metadata-completeness-and-vlo-facets}

Providing correct and complete metadata improves discoverability. The
more information you provide about your resources and their context, the
higher the expected recall for queries that match your data. If you are
looking for good representation and discoverability through the VLO, pay
attention to its \textbf{facets}, which can be considered one of the
main entry points into the data aggregated into the VLO. Ideally
\emph{each metadata record} covers as many facets as possible.
Furthermore, a number of non-facet fields are displayed prominently in
the search results, in particular a record's \textbf{title and
description}, as well as licence and availability information. Records
that are lacking this essential information not only are harder to find
that those which do, but also are less likely to receive a high ranking
in the search results. \textgreater{} {[}sh{]} ``not only are harder to
find \textbf{than} those which do, but \textbf{are also} less likely''

A repository may simply export its metadata to \textbf{Dublin Core}.
This is one of the requirements of OAI-PMH (see
\href{https://www.openarchives.org/OAI/2.0/openarchivesprotocol.2015-01-08.htm\#Record}).
However, one has to be aware that the set of attributes included in
`pure' Dublin Core does not necessarily allow for complete or
semantically unambiguous metadata. If you are looking to tune your
metadata to one ore more parts of the CLARIN infrastructure, it's
strongly suggested to investigate the option of exporting to a fitting
CMDI profile, or to make your own dedicated profile if no such profile
can be found. If this is not an option, you can also consider using the
\href{http://www.language-archives.org/OLAC/metadata.html}{OLAC
extensions} to enrich your Dublin Core metadata.

Finally, be aware that the VLO, and the CLARIN infrastructure in
general, will generally assume that a metadata record is
self-sufficient. This might differ from the way `native' repositories
process and present metadata. For example, if the name of the identity
of your organisation is not represented in a record, it will also not be
associated with the described resources by the VLO. A metadata hierarchy
might provide additional context, such as project information,
collection details or legal status, but this will not cause this
information to be presented in the context of the resource in a clear
way unless it is explicitly included in its own metadata record.

To find out how well your metadata record will be represented in the VLO
you may consult the \textbf{Metadata Curation Module}. Here, you are
able to upload your metadata record, browse through a list of available
profiles or upload a profile URL in order to obtain a ``score'' and an
overview of facet coverage for the respective profile or file.

\subsubsection{Values and semantics}\label{values-and-semantics}

{[}Consider given vocabularies, e.g.~resource type (including links
where to find these){]}

{[}Ensure conformity among records of a collection/institution
(e.g.~naming conventions){]}

Facet ``collection'': If you are preparing a set of records that belong
to one collection, this collection can be specified in the CMDI header
in each record and thus made findable via the VLO facet. If so, make
sure that the collection is named homogeneously accross records so that
the collection name will appear only once in the VLO collection facet.
{[}Prefix collection{]}

{[}Modeller's responsibility to avoid ambiguities that can't be resolved
technically (e.g. \textless{}date\textgreater{} in very generally named
branches: date of what (creation/description/submission/modification)?):
naming and most importantly concept/context{]}

{[}Avoid non-alphanumeric characters in titles
(\href{https://www.gitbook.com/book/cmdi-taskforce/cmdi-best-practices/discussions/19}{Discussion
\#19}){]}

\subsubsection{Links and context}\label{links-and-context}

{[}Importance of landing page{]}

\end{document}
