\documentclass[]{article}
\usepackage{lmodern}
\usepackage{amssymb,amsmath}
\usepackage{ifxetex,ifluatex}
\usepackage{fixltx2e} % provides \textsubscript
\ifnum 0\ifxetex 1\fi\ifluatex 1\fi=0 % if pdftex
  \usepackage[T1]{fontenc}
  \usepackage[utf8]{inputenc}
\else % if luatex or xelatex
  \ifxetex
    \usepackage{mathspec}
  \else
    \usepackage{fontspec}
  \fi
  \defaultfontfeatures{Ligatures=TeX,Scale=MatchLowercase}
\fi
% use upquote if available, for straight quotes in verbatim environments
\IfFileExists{upquote.sty}{\usepackage{upquote}}{}
% use microtype if available
\IfFileExists{microtype.sty}{%
\usepackage{microtype}
\UseMicrotypeSet[protrusion]{basicmath} % disable protrusion for tt fonts
}{}
\usepackage{hyperref}
\hypersetup{unicode=true,
            pdfborder={0 0 0},
            breaklinks=true}
\urlstyle{same}  % don't use monospace font for urls
\IfFileExists{parskip.sty}{%
\usepackage{parskip}
}{% else
\setlength{\parindent}{0pt}
\setlength{\parskip}{6pt plus 2pt minus 1pt}
}
\setlength{\emergencystretch}{3em}  % prevent overfull lines
\providecommand{\tightlist}{%
  \setlength{\itemsep}{0pt}\setlength{\parskip}{0pt}}
\setcounter{secnumdepth}{0}
% Redefines (sub)paragraphs to behave more like sections
\ifx\paragraph\undefined\else
\let\oldparagraph\paragraph
\renewcommand{\paragraph}[1]{\oldparagraph{#1}\mbox{}}
\fi
\ifx\subparagraph\undefined\else
\let\oldsubparagraph\subparagraph
\renewcommand{\subparagraph}[1]{\oldsubparagraph{#1}\mbox{}}
\fi

\date{}

\begin{document}

\subsection{Multilingual metadata}\label{multilingual-metadata}

Multilinguality should be considered for components and elements. If a
language other than English is allowed/possible, an English version
should be provided wherever feasible. A content language tag
(\texttt{@xml:lang} attribute) should be used to determine the language
of the respective metadata item. The choice of the language tag should
be based on the
\href{https://www.w3.org/International/questions/qa-choosing-language-tags}{W3C
recommendations on this topic}, i.e.~use an
\href{https://tools.ietf.org/rfc/bcp/bcp47.txt}{BCP 47}
\href{https://www.iana.org/assignments/language-subtag-registry/language-subtag-registry}{language
tag} (see also
\href{/authoring_component_metadata_records/the_component_section.md\#cs3}{CS3}).

All elements allowing for values that do not come from an (implicit or
explicit) vocabulary are good candidates for being multilingual.
Especially titles and descriptions should be multilingual. Provide
English variants first, or if not feasible start with the `native'
language followed by English. Set the content language tag
(\texttt{@xml:lang} attribute) for all variants, including English
and/or the `native' language, i.e., do not have an multilingual element
without an \texttt{@xml:lang} attribute. Also do not allow multiple
variants for a language, e.g., there should be only one English or
native variant.

When an external vocabulary is used they should provide at least an
english label, but they might provide additional labels in other
languages. When the label is also the value, the english label should be
preferred, unless the (local) primary audience of the record is native.
Renderers of the record can look up the additional labels to deal with a
multilingual setup.

\end{document}
