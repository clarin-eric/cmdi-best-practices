\subsection{Dealing with uncertainty}\label{dealing-with-uncertainty}

In many cases one has to model for uncertainty. For example, the
creation date of a document may not be known or only by approximation.
Rather than making a field for such information optional, it might be
desirable to provide the metadata creator with an option to make this
uncertainty explicit. A ``string'' typed field provides this option
(i.e.~it allows the metadata creator to enter ``?'' or ``unknown'') but
does so at the cost of value ``safety'' - in this example it does not
enforce a standardised date format. A better approach is to require a
value of a more specific type if applicable, for example a date or
integer. Alternatively a vocabulary or pattern can be used to constrain
the values that may be entered, optionally including a literal value
like `Unknown'. You can also add an (optional) attribute with type
``boolean'' or a vocabulary to allow the metadata creator to specify the
level of uncertainty.

\begin{verbatim}
<cmdp:birthYear certainty="uncertain">1180</cmdp:birthYear>
\end{verbatim}

Note that in many cases making an element optional is a valid modelling
decision for cases where information might not be available. However,
preferably this would be reserved for the element being not applicable.

\begin{verbatim}
<cmdp:birthYear confidence="not found"/>
\end{verbatim}

Notice that this provides more information than leaving out the unknown
birth year. The confidence value tells us that one has been looking for
the birth year but it has not been found yet. Leaving out the birth year
would mean in this case that its unknown, but has also not been
researched.
